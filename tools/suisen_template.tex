\documentclass[a4paper,10pt]{jarticle}

\usepackage{longtable}
\usepackage[top=20truemm,bottom=0truemm,left=15truemm,right=15truemm]{geometry}

\def\GM{千葉大学文学部・伝 康晴}
\def\ORDER{XX}
\def\NOTE{招待講演を除く}
%\def\NOTE{1日目.招待講演を除く} % 2日開催で分割する場合
%\def\NOTE{2日目.招待講演を除く} % 2日開催で分割する場合

\begin{document}
\pagestyle{empty}

\begin{flushright}
人工知能学会 言語・音声理解と対話処理研究会\\
主査 \GM
\end{flushright}


\begin{center}
\large \bf
優秀論文推薦のお願い	
\end{center}


今回は、第{\ORDER}回言語・音声理解と対話処理研究会にご参加いただき、ありがとうございます。本研究会は、広く言語、音声、対話の研究分野を活性化することを目的の1つとして活動をしております。この活性化の一環として、奨励的な意味合いを含めた優秀論文賞を年度ごとに選定してきました。この選考にあたっては、研究会にご参加いただいているみなさまからのご意見を大いに参考にさせていただいております。
つきましては、今回の発表の中から優秀論文賞に推薦してよいと思うものに○をつけてください(招待講演は除きます。複数回答可)。よかったと思うものに積極的に○をつけてください。
お手数をおかけしますが、ご協力をお願いいたします。

\begin{table}[h]
\begin{tabular}{p{17.5cm}}
\hline
\\
お名前: \\
\\
ご所属(学生さんの場合は学年も):\\
\\
連絡先メールアドレス:\\
\\
\hline
\end{tabular}
\end{table}

\begin{center}
第{\ORDER}回 人工知能学会 言語・音声理解と対話処理研究会プログラム(\NOTE)	
\end{center}

%\small %% if too many to fit in 2 pages
%\renewcommand{\arraystretch}{1.28}

\input{candidates} % assumes longtable input such as:
%% \begin{longtable}[c]{|p{1.5cm}|p{15cm}|}
%% \hline
%%  推薦する & 演題 \\\endfirsthead
%% \hline
%%  推薦する & 演題 \\\endhead
%% \hline
%% &1.【若手】相槌・フィラー予測とのマルチタスク学習によるターンテイキング予測\\
%% &原康平・井上昂治・高梨克也・河原達也(京大)\\
%% \hline
%% &2.単語の分散表現を用いた相槌生成タイミングの予測\\
%% &黒田和矢・狩野芳伸(静岡大)\\
%% \hline
%% &3.【若手】初対面対話における好感のモデリングと発話構成要素の選択\\
%% &田中滉己・井上昂治・中村静・高梨克也・河原達也(京大)\\
%% \hline
%% \end{longtable}
\end{document}


